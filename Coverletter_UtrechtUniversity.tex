\documentclass[11.8pt,a4paper,sans]{moderncv}
\moderncvstyle{banking}
\moderncvcolor{blue}

\usepackage[utf8]{inputenc}
\usepackage{enumitem}
\usepackage[scale=0.75]{geometry}
\AtBeginDocument{
  \hypersetup{
    colorlinks=true,
    linkcolor=blue,
    urlcolor=blue,
    citecolor=blue,
    filecolor=blue,
    menucolor=blue,
    runcolor=blue,
    linkbordercolor=blue,
    pdfborderstyle={/S/U/W 1} % sets underlining
}

}

\name{Krishna}{Narasimhan}

\begin{document}

\recipient{Prof. Fabiano Dalpiaz}{Utrecht University\\Utrecht, The Netherlands}
\date{\today}
\opening{Dear Prof. Fabiano Dalpiaz}
\closing{Sincerely,}

\makelettertitle

In this letter, I am expressing my keen interest in joining Utrecht University as an Assistant Professor (Education Track) in Software Engineering. My expertise in Software systems, with a focus on program analysis, model based tooling and DSL technology, aligns perfectly with the requirements of this role. I will highlight four key achievements at the end, referencing them throughout to clearly demonstrate my fit for the position.

Holding a Ph.D. in Computer Science from Goethe University, my skills in software analysis and tool development are well-established. My doctoral research pioneered new methods for automated code refactoring and data representation migration in large software systems, leading to publications in top-tier software engineering platforms like ASE, PEPM, and JASE. My work in code clone refactoring was honored with the ACM SIGSOFT Distinguished Paper Award (see:\href{https://ase-conferences.org/olbib/}{ASE best papers}) - \texttt{[A1*]}.

As a Postdoctoral Researcher at Technische Universität Darmstadt, my work spans multiple research areas, including API misuses, software cybersecurity, and AI-based coding assistance. I managed a team of Ph.D. students and student helpers to investigate and improve the state of the art in AI-based coding assistance. My research responsibilities also included maintaining the CogniCrypt project and advancing the state of the art in API misuse detection and software security. My contributions here have been published in various top-tier software engineering venues. I actively maintained several national and international collaborations, including with the University of Paderborn, Germany,  the University of Brasilia, Brazil, and Nanjing University, China. I also have experience acquiring external funding, having helped author the bulk of two successful grant proposals to the DFG. , I gained invaluable experience in administrative roles, including serving on hiring committees and directing academic programs (see: (\href[]{https://www.crossing.tu-darmstadt.de/outreach/female_student_mentoring_and_networking/index.en.jsp}{Female Student Mentoring program} and (\href[]{https://www.crossing.tu-darmstadt.de/crc_1119/people_crossing/directorate/index.en.jsp}{Crossing SFB Directorate})) - \texttt{[A2, A4*]}).

I have honed my didactic skills by teaching various courses as an educator. I designed a lecture on concepts of programming languages, prepared the slides, recorded videos, and offered the course multiple times, both in person and online. I designed and taught a unique course about using AI in coding support.  Students are graded on a scientific vision paper they must submit. You can visit the course website . Many students' work from this course has been accepted for publication at international venues, such as CAIN and AISTA. The courses, including lecture recordings, are accessible online (see: \href[]{https://www.youtube.com/playlist?list=PLjkB5k_lfPhTNUVNiGb_PeuioC-XV5kS5}{TU Darmstadt's STG lecture channel}, \href[]{https://stg-tud.github.io/AI4CI/}{Ai4CI course website}).  I currently serve as a part-time lecturer at Hochschule Darmstadt, where I assist newly enrolled Bachelor's students in grasping the fundamentals of programming. (see: \href{https://splan.eit.h-da.de/s/S_WIng_BA_1BWIng.htm}{HDa Lecturer Assignment}) - \texttt{[A3*]}

My industry experience as an IT Consultant/Software Architect at Itemis AG involved close collaboration with key clients like Canon and Siemens, focusing on domain-specific languages and modeling tools. Currently, at AIQ, I am leading product research and development in cloud-based AI systems (see: [\href{https://aiqualityhub.com/en/about-us/}{About Us at AIQ}]), , and organizing major events like the \href[]{https://www.linkedin.com/feed/update/urn:li:activity:7141080318614073344/}{AI Quality summit} which incorporated speakers and guests from public, private and regulatory sectors. Furthermore, my position entails hands-on engagement in development processes, a vital aspect of my contribution to the dynamic and evolving environment of a startup company. This blend of academic and practical experience positions me ideally for a role that values societal and industrial impact, a philosophy that I believe resonates with the Software Engineering and Information Systems group at Utrecht University.

\section*{Accomplishments}
\begin{enumerate}[label=A\arabic*]
\item Authored a distinguished paper that formed the cornerstone of my PhD dissertation, demonstrating a significant contribution to the field.
\item Elected to various leadership and mentoring roles, including positions in the Directorate of Crossing project and a mentoring program for female students, showcasing my commitment to diversity and leadership in academia.
\item Designed and delivered innovative, interactive courses with real-world relevance. My 'Concepts of Programming Languages' course, where students developed multiple programming languages from scratch, and the 'AI4CA' seminar, which led to several student publications in peer-reviewed conferences and workshops, have been particularly impactful. This success is underscored by positive student feedback; for instance, the 2021 'Concepts of Programming Languages' course received accolades for respectful student engagement.
\item Engaged in national and international collaborations (including Brazil and China) resulting in publications at prestigious conferences and involvement in collaborative DFG-funded projects, highlighting my global research footprint.
\end{enumerate}


 I appreciate your consideration of my application and look forward to discussing my candidacy further.

 \makeletterclosing

\end{document}