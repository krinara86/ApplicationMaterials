\documentclass[11.8pt,a4paper,sans]{moderncv}
\moderncvstyle{banking}
\moderncvcolor{blue}

\usepackage[utf8]{inputenc}
\usepackage{enumitem}
\usepackage[scale=0.75]{geometry}
\AtBeginDocument{
  \hypersetup{
    colorlinks=true,
    linkcolor=blue,
    urlcolor=blue,
    citecolor=blue,
    filecolor=blue,
    menucolor=blue,
    runcolor=blue,
    linkbordercolor=blue,
    pdfborderstyle={/S/U/W 1} % sets underlining
}

}

\name{Krishna}{Narasimhan}

\begin{document}

\recipient{Prof. Fabiano Dalpiaz}{Utrecht University\\Utrecht, The Netherlands}
\date{\today}
\opening{Dear Prof. Fabiano Dalpiaz}
\closing{Sincerely,}

\makelettertitle

In this letter, I express my earnest interest in the position of Assistant Professor (Education Track) in Software Engineering at Utrecht University. My background as a computer scientist, with a substantial emphasis on software technology, directly corresponds to the demands of this role. For the purpose of coherent presentation and clarity, this cover letter concludes with an enumeration of four principal accomplishments. These are subsequently referenced throughout the text, ensuring a contextually grounded and methodical exposition of my suitability for the position.

Having earned a Ph.D. in Computer Science from Goethe University, I have developed comprehensive software analysis and tooling expertise. My doctoral research focused on developing novel techniques for the automated refactoring of code clones and automatic migration of data representations in large-scale software systems. The resulting works have been published in top-tier software engineering venues, including the Automated Conference on Software Engineering (ASE), Partial Evaluation and Program Manipulation (PEPM), and the Journal of Automated Software Engineering (JASE). I have also been awarded the prestigious \textit{ACM SIGSOFT Distinguished Paper Award} for my work on code clone refactoring - \href{https://ase-conferences.org/olbib/}{Link to ASE best papers}. - \texttt{[A1*]}.

As a Postdoctoral Researcher at Technische Universität Darmstadt, my work spans multiple research areas, including API misuses, software cybersecurity, and AI-based coding assistance. I managed a team of Ph.D. students and student helpers to investigate and improve the state of the art in AI-based coding assistance. My research responsibilities also included maintaining the CogniCrypt project and advancing the state of the art in API misuse detection and software security. My contributions here have been published in various top-tier software engineering venues. I actively maintained several national and international collaborations, including with the University of Paderborn, Germany,  the University of Brasilia, Brazil, and Nanjing University, China. I also have experience acquiring external funding, having helped author the bulk of two successful grant proposals to the DFG. In terms of organizational and managerial activities, my experience serving on the hiring committee at Technische Universität Darmstadt, the directorate of Crossing SFB (\href[]{https://www.crossing.tu-darmstadt.de/crc_1119/people_crossing/directorate/index.en.jsp}{Link}) and the female student mentoring program (\href[]{https://www.crossing.tu-darmstadt.de/outreach/female_student_mentoring_and_networking/index.en.jsp}{Link}) have provided me with valuable insights into the administrative processes in an academic setting.  - \texttt{[A2, A4*]}.

I have honed my didactic skills by teaching various courses as an educator. I designed a lecture on concepts of programming languages, prepared the slides, recorded videos, and offered the course multiple times, both in person and online. The lectures are available online at the \href[]{https://www.youtube.com/playlist?list=PLjkB5k_lfPhTNUVNiGb_PeuioC-XV5kS5}{TU Darmstadt's STG lecture channel}. I've created and taught a unique course about using AI in coding support. Students are graded on a scientific vision paper they must submit. You can visit the course website \href[]{https://stg-tud.github.io/AI4CI/}{here}. Many students' work from this course has been accepted for publication at international venues, such as CAIN and AISTA. You can visit the course website \href[]{https://splan.eit.h-da.de/s/S_WIng_BA_1BWIng.htm}{here} - \texttt{[A3*]}

In my previous capacity as an IT Consultant/Software Architect at Itemis AG, I established a noteworthy record in industrial software development and client management. This role involved intricate collaboration with clients in the automotive and embedded systems sectors, notably with Canon in Eindhoven, Netherlands, and Siemens in Chennai, India. My primary focus was the development of tailored domain-specific languages and modeling tools. Currently, at the forefront of technological endeavors at AIQ (see [\href{https://aiqualityhub.com/en/about-us/}{About Us}]), my responsibilities encompass spearheading product research, designing, and supervising the construction of cloud-based AI systems and prototypes, alongside managing an international team. Moreover, my role includes direct involvement in development processes. This practical industry experience has significantly bolstered my capacity to produce research outcomes that are not only industrially relevant but also beneficial to broader society. I perceive a similar emphasis on industry and societal impact within the Software Engineering and Information Systems group at Utrecht University.

\section*{Accomplishments}
\begin{enumerate}[label=A\arabic*]
\item Authored a distinguished paper that formed the cornerstone of my PhD dissertation, demonstrating a significant contribution to the field.
\item Elected to various leadership and mentoring roles, including positions in the Directorate of Crossing project and a mentoring program for female students, showcasing my commitment to diversity and leadership in academia.
\item Designed and delivered innovative, interactive courses with real-world relevance. My 'Concepts of Programming Languages' course, where students developed multiple programming languages from scratch, and the 'AI4CA' seminar, which led to several student publications in peer-reviewed conferences and workshops, have been particularly impactful. This success is underscored by positive student feedback; for instance, the 2021 'Concepts of Programming Languages' course received accolades for respectful student engagement.
\item Engaged in national and international collaborations (including Brazil and China) resulting in publications at prestigious conferences and involvement in collaborative DFG-funded projects, highlighting my global research footprint.
\end{enumerate}


 I appreciate your consideration of my application and look forward to discussing my candidacy further.

 \makeletterclosing

\end{document}