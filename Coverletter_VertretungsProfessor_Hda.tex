\documentclass[12pt,a4paper,sans]{moderncv}
\moderncvstyle{banking}
\moderncvcolor{blue}

\usepackage[utf8]{inputenc}
\usepackage[scale=0.75]{geometry}
\AtBeginDocument{
  \hypersetup{
    colorlinks=true,
    linkcolor=blue,
    filecolor=magenta,      
    urlcolor=cyan,
}
}

\name{Krishna}{Narasimhan}

\begin{document}

\recipient{Prof. Sven-Bodo Scholz}{Radboud University\\Nijmegen, The Netherlands}
\date{\today}
\opening{Dear Prof. Sven-Bodo Scholz}
\closing{Sincerely,}

\makelettertitle

I am writing to express my profound interest in the VertretungsProfessor position at Hochschule Darmstadt. I am an enthusiastic computer scientist with a strong background in software technology, and my experience and skills closely align with the requirements of this role.

Having earned a Ph.D. in Computer Science from Goethe University, I have developed comprehensive code analysis and transformation expertise. My doctoral research focused on developing novel techniques for the automated refactoring of code clones and automatic migration of data representations in large-scale software systems. The resulting works have been published in top-tier software engineering venues, including the Automated Conference on Software Engineering (ASE), Partial Evaluation and Program Manipulation (PEPM), and the Journal of Automated Software Engineering (JASE). I have also been awarded the prestigious ACM SIGSOFT Distinguished Paper Award for my work on code clone refactoring - \href{https://ase-conferences.org/olbib/}{Link to ASE best papers}.

As a Postdoctoral Researcher at Technische Universität Darmstadt, my work spans multiple research areas, including API misuses, software cybersecurity, and AI-based coding assistance. I manage a team of Ph.D. students and student helpers to investigate and improve the state of the art in AI-based coding assistance. My research responsibilities also include maintaining the CogniCrypt project and advancing the state of the art in API misuse detection and software security. My contributions here have been published in various top-tier software engineering venues. I actively maintain several national and international collaborations, including with the University of Paderborn, Germany,  the University of Brasilia, Brazil, and Nanjing University, China. I also have experience acquiring external funding, having helped author the bulk of two successful grant proposals to the DFG. In terms of organizational and managerial activities, my experience serving on the hiring committee at Technische Universität Darmstadt, the directorate of Crossing SFB (\href[]{https://www.crossing.tu-darmstadt.de/crc_1119/people_crossing/directorate/index.en.jsp}{Link}) and the female student mentoring program (\href[]{https://www.crossing.tu-darmstadt.de/outreach/female_student_mentoring_and_networking/index.en.jsp}{Link}) have provided me with valuable insights into the administrative processes in an academic setting. 

I have honed my didactic skills by teaching various courses as an educator. I designed a lecture on concepts of programming languages, prepared the slides, recorded videos, and offered the course multiple times, both in person and online. The lectures are available online at the \href[]{https://www.youtube.com/playlist?list=PLjkB5k_lfPhTNUVNiGb_PeuioC-XV5kS5}{TU Darmstadt's STG lecture channel}. I've created and currently teach a unique course about using AI in coding support. Students are graded on a scientific vision paper they must submit. You can visit the course website \href[]{https://stg-tud.github.io/AI4CI/}{here}. Many students' work from this course has been accepted for publication at international venues, such as CAIN and AISTA.

I have a successful track record of industrial software development and client management, as apparent from my previous role as an IT Consultant/Software Architect at Itemis AG. I worked closely with automotive and embedded systems clients (specifically Canon in Eindhoven, Netherlands, and Siemens in Chennai, India) to design bespoke domain-specific languages and modeling tools. This hands-on industry experience enhanced my ability to generate research results beneficial to industry and society, a focus I believe is shared by the Software Engineering group at Hochschule Darmstadt.

In setting up my research line, I have a proven record of leading research teams, being a significant part of acquiring external funding, and successfully pursuing novel research directions. I am passionate about the intersection of traditional software development and innovative AI technologies.

I am thrilled at the prospect of contributing to Hochschule Darmstadt's pioneering work in software engineering. I appreciate your consideration of my application and look forward to discussing my candidacy further.

\makeletterclosing

\end{document}