\documentclass[12pt,a4paper,sans]{moderncv} 
\moderncvstyle{banking} 
\moderncvcolor{blue} 
\usepackage{fontawesome5}
\usepackage[scale=0.75]{geometry}
\usepackage{enumitem}

\AtBeginDocument{
  \hypersetup{
    colorlinks=true,
    linkcolor=blue,
    filecolor=magenta,      
    urlcolor=cyan,
}
}


\name{Dr. Krishna}{Narasimhan}
\phone[mobile]{+49~1741750063}
\email{krishna.nm86@gmail.com}
\renewcommand*{\emailsymbol}{{\small\faEnvelope}}
\social[Linkedin][www.linkedin.com/in/krishna-narasimhan-41a8a214/]{\faLinkedin \hspace{0.1cm} LinkedIn}
%\photo[64pt][0.4pt]{krishna.png}

\begin{document}
\pagestyle{empty}
\makecvtitle

\section{Profile}
Experienced Software Engineer and Computer Scientist with a diverse portfolio that encompasses managing industry-standard software products, collaborating with international client teams, teaching graduate level courses and spearheading research and development efforts. Notably adept at enhancing productivity through developing tools primarily aimed at assisting other software engineers and domain experts. Academic and professional interests include developer tools, static/dynamic program analysis, domain-specific languages, and AI for coding assistance. A committed educator and leader with a demonstrated history of leading research teams. Passionate about the intersection of traditional software development and innovative AI technologies.


\section{Employment}
\cventry{Jan 2020 -- present}{Postdoctoral Researcher}{Technische Universität Darmstadt}{Darmstadt, Germany}{}{
  As the lead of a research team, I design and build AI-based code models that are reliable, explainable, and capable of understanding software semantics. My investigations extend to advancing the current standard in API misuse detection and program security. In addition to these responsibilities, I educate graduate students on programming languages and AI coding assistance, while also serving on the University's Hiring Committee (Berufskomission).
}
\vspace{1ex}
\cventry{Jan 2017 -- Jan 2020}{IT Consultant/Software Architect}{Itemis AG}{Stuttgart, Germany}{}{
  As a Software Architect and consultant, I designed and developed bespoke domain-specific languages and software tools, primarily catering to clients in the automotive and embedded systems industries. This involved understanding their unique needs, crafting precise language specifications, and building user-friendly tools to improve their operational efficiency and project effectiveness.
}
\vspace{1ex}
\cventry{September 2013 -- Dec 2016}{Research/Teaching Assistant}{Goethe University}{Frankfurt, Germany}{}{Managed the Foundations of Programming Languages lecture series while conducting research on addressing productivity issues associated with evolving software.}
\vspace{1ex}
\cventry{September 2011 -- Dec 2013}{Research/Teaching Assistant}{Universität des Saarlandes}{Saarbrücken, Germany}{}{Contributed to the Software Engineering Chair as a research assistant, supported graduate-level courses on software engineering and security, and aided the team in developing advanced program analysis tools for Android Malware detection.}
\vspace{1ex}
\cventry{September 2009 -- Dec 2010}{Software Development Engineer}{Scantron}{Chennai, India}{}{Developed and maintained a diverse software suite for the education sector, utilizing languages such as Script\#, SQL, and JavaScript. Collaborated with a team of developers and testers to deliver high-quality software products.}
\vspace{1ex}
\section{Education}
\cventry{2014--2017}{Ph.D. in Computer Software}{Goethe University}{Frankfurt, Germany}{Suma cum laude}{\raggedright\textbf{Thesis} - \texttt{Combining user interaction and automation to evolve source code.}}
\vspace{0.5ex}
\cventry{2011--2014}{Master of Science - MSc, Computer Science}{Universität des Saarlandes}{Saarbrücken, Germany}{}{\raggedright\textbf{Thesis} - \texttt{Android decompression chamber:A hybrid(static/dynamic) approach to detecting Android malware.}}
\vspace{0.5ex}
\cventry{2007--2010}{Master of Computer Applications - MCA}{Anna University}{Chennai, India}{}{}
\cventry{2004--2007}{Bachelor of Science - BSc, Mathematics}{University of Madras}{Chennai, India}{}{}

\section{Service}
I have served on the program committee of several top-tier software engineering conferences including \href{https://2021.ecoop.org/committee/ecoop-2021-ecoop-artifacts-artifact-evaluation-committee}{ECOOP}, \href{https://conf.researchr.org/committee/sle-2019/sle-2019-papers-artifact-evaluation-committee}{SLE}, \href{https://conf.researchr.org/committee/issta-2021/issta-2021-artifact-evaluation-artifact-evaluation-committee}{ISSTA}, and \href{https://2020.splashcon.org/committee/splash-2020-Artifacts-artifact-evaluation-committee}{OOPSLA}.


\section{Projects}
\cvitem{CogniCrypt}{State of the art in allow-listing based API misuse detection. \href{https://ci.eclipse.org/cognicrypt/}{Website}}
\cvitem{jGuard}{Java extensions Making APIs misuse resilient by design. \href{https://youtu.be/tnQGMyMjZRA}{Video demonstration }}
\cvitem{mbeddr}{an extensible set of integrated languages for embedded software development - \href{http://www.mbeddr.com}{Website}}
\cvitem{Transparent abstractions}{Effective methods to simplify software reuse - \href{https://fileadmin.cs.lth.se/cs/research/sde/transparent-abstractions/}{Website}}

\section{Skills}
\cvitem{Technical skills}{
Experienced in object oriented programming languages and building domain-specific languages/ modelling tools using XText and Jetbrains MPS. Skilled in static/dynamic program analysis and tooling for code assistance. Proficient in software design, maintenance, and security.
}
\cvitem{Soft skills}{
Experienced in technical writing, presenting, and leadership. Strong collaborator with international and inter-disciplinary teams. Effective educator at graduate level. Excellent communicator with superior problem-solving abilities.
}
\section{PUBLICATIONS}


    \begin{itemize}
        \item \textbf{Software tooling}
        \begin{itemize}

\item \emph{Fex: Assisted Identification of Domain Features from C Programs} \\
Patrick Müller, \textbf{Krishna Narasimhan}, and Mira Mezini \\
In 21st IEEE International Working Conference on Source Code Analysis and Manipulation, SCAM 2021, IEEE, 2021, pp. 170--180 \\
\url{https://doi.org/10.1109/SCAM52516.2021.00029}

\item \emph{Cleaning up copy-paste clones with interactive merging} \\
\textbf{Krishna Narasimhan}, Christoph Reichenbach, and Julia Lawall \\
In Autom. Softw. Eng., volume 25, number 3, 2018, pp. 627--673 \\
\url{https://doi.org/10.1007/s10515-018-0238-5}

\item \emph{Combining user-interaction and automation to evolve source code} \\
\textbf{Krishna Narasimhan} \\
PhD Thesis, Goethe University Frankfurt, Frankfurt am Main, Germany, 2017 \\
\url{http://publikationen.ub.uni-frankfurt.de/frontdoor/index/index/docId/42783}

\item \emph{Interactive data representation migration: exploiting program dependence to aid program transformation} \\
\textbf{Krishna Narasimhan}, Christoph Reichenbach, and Julia Lawall \\
In Proceedings of the 2017 ACM SIGPLAN Workshop on Partial Evaluation and Program Manipulation, PEPM 2017, ACM, 2017, pp. 47--58 \\
\url{https://doi.org/10.1145/3018882.3018890}


\item \emph{Copy and Paste Redeemed (T)} \\
\textbf{Krishna Narasimhan} and Christoph Reichenbach \\
In 30th IEEE/ACM International Conference on Automated Software Engineering, ASE 2015, IEEE Computer Society, 2015, pp. 630--640 \\
\url{https://doi.org/10.1109/ASE.2015.39}

\item \emph{Clone Merge - An Eclipse Plugin to Abstract Near-Clone C++ Methods} \\
\textbf{Krishna Narasimhan} \\
In 30th IEEE/ACM International Conference on Automated Software Engineering, ASE 2015, IEEE Computer Society, 2015, pp. 819--823 \\
\url{https://doi.org/10.1109/ASE.2015.103}
\end{itemize}


\item \textbf{API misuse}
\begin{itemize}

\item \emph{FUM: A Framework for API Usage Constraint and Misuse Classification} \\
Michael Schlichtig, Steffen Sassalla, \textbf{Krishna Narasimhan}, and Eric Bodden \\
In  2022 IEEE International Conference on Software Analysis, Evolution and Reengineering (SANER), 15.-18. March 2022, Honolulu, HI, USA,  IEEE, 2022 \\
\url{https://doi.org/10.1109/SANER53432.2022.00085}

\item \emph{jGuard: Programming Misuse-Resilient APIs} \
Simon Binder, \textbf{Krishna Narasimhan}, Svenja Kernig, Mira Mezini \
In Proceedings of the 15th ACM SIGPLAN International Conference on Software Language Engineering, SLE 2022, Auckland, New Zealand, December 6-7, 2022, ACM, 2022 \
\url{https://doi.org/10.1145/3567512.3567526}

\item \emph{Dealing with Variability in API Misuse Specification} \\
Rodrigo Bonifácio, Stefan Krüger, \textbf{Krishna Narasimhan}, Eric Bodden, Mira Mezini \\
In 35th European Conference on Object-Oriented Programming, ECOOP 

2021, Aarhus, Denmark, July 11-17, 2021, Schloss Dagstuhl - Leibniz-Zentrum für Informatik, 2021 \\
\url{https://doi.org/10.4230/LIPIcs.ECOOP.2021.19}

\item \emph{BRAID: an API recommender supporting implicit user feedback} \\
Yu Zhou, Haonan Jin, Xinying Yang, Taolue Chen, \textbf{Krishna Narasimhan}, and Harald C. Gall \\
In ESEC/FSE '21: 29th ACM Joint European Software Engineering Conference and Symposium on the Foundations of Software Engineering, ACM, 2021, pp. 1510--1514 \\
\url{https://doi.org/10.1145/3468264.3473111}
\end{itemize}
\item \textbf{AI and programming}

\begin{itemize}
    \item \emph{Evaluating and improving transformers pre-trained on ASTs for Code Completion} \\
Marcel Ochs, \textbf{Krishna Narasimhan}, and Mira Mezini \\
In IEEE International Conference on Software Analysis, Evolution and Reengineering, SANER 2023, Taipa, Macao, March 21-24, 2023, IEEE, 2023 \\
\url{https://doi.org/10.1109/SANER56733.2023.00096}

\item \emph{Towards Code Generation from BDD Test Case Specifications: A Vision} \\
Leon Chemnitz, David Reichenbach, Hani Aldebes, Mariam Naveed, \textbf{Krishna Narasimhan}, and Mira Mezini \\
In CoRR, abs/2305.11619, 2023 \\
\url{https://doi.org/10.48550/arXiv.2305.11619}

\item \emph{Impact of programming languages on machine learning bugs} \\
Sebastian Sztwiertnia, Maximilian Grübel, Amine Chouchane, Daniel Sokolowski, \textbf{Krishna Narasimhan}, Mira Mezini \\
In AISTA 2021: Proceedings of the 1st ACM International Workshop on AI and Software Testing/Analysis, Virtual Event, Denmark, July 12, 2021, ACM, 2021 \\
\url{https://doi.org/10.1145/3464968.3468408}

\item \emph{NerdBug: automated bug detection in neural networks} \\
Foad Jafarinejad, \textbf{Krishna Narasimhan}, and Mira Mezini \\
In AISTA 2021: Proceedings of the 1st ACM International Workshop on AI and Software Testing/Analysis, ACM, 2021, pp. 13--16 \\
\url{https://doi.org/10.1145/3464968.3468409}



\end{itemize}

\item \textbf{Software security}
\begin{itemize}

\item \emph{Exploring the use of static and dynamic analysis to improve the performance of the mining sandbox approach for android malware identification} \\
Francisco Handrick da Costa, Ismael Medeiros, Thales Menezes, João Victor da Silva, Ingrid Lorraine da Silva, Rodrigo Bonifácio, \textbf{Krishna Narasimhan}, and Márcio Ribeiro \\
In Journal of Systems and Software, 183, 111092, 2022 \\
\url{https://doi.org/10.1016/j.jss.2021.111092}

\item \emph{To Fix or Not to Fix: A Critical Study of Crypto-misuses in the Wild} \\
Anna-Katharina Wickert, Lars Baumgärtner, Michael Schlichtig, \textbf{Krishna Narasimhan}, Mira Mezini \\
In IEEE International Conference on Trust, Security and Privacy in Computing and Communications, TrustCom 2022, Wuhan, China, December 9-11, 2022, IEEE, 2022 \\
\url{https://doi.org/10.1109/TrustCom56396.2022.00051}
\end{itemize}

\end{itemize}
\end{document}
