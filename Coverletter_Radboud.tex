\documentclass[12pt,a4paper,sans]{moderncv}
\moderncvstyle{banking}
\moderncvcolor{blue}

\usepackage[utf8]{inputenc}
\usepackage[scale=0.75]{geometry}
\AtBeginDocument{
  \hypersetup{
    colorlinks=true,
    linkcolor=blue,
    filecolor=magenta,      
    urlcolor=cyan,
}
}

\name{Krishna}{Narasimhan}

\begin{document}

\recipient{Prof. Sven-Bodo Scholz}{Radboud University\\Nijmegen, The Netherlands}
\date{\today}
\opening{Dear Prof. Sven-Bodo Scholz}
\closing{Sincerely,}

\makelettertitle

I am writing to express my profound interest in the Assistant/Associate of Software Technology position at Radboud University. I am an enthusiastic computer scientist with a strong background in software technology and am confident that my experience and skills closely align with the requirements of this role.

Having earned a Ph.D. in Computer Software from Goethe University, I have developed broad expertise in code analysis and transformation. My doctoral research focused on developing novel techniques for the automated refactoring of code clones and automatic migration of data representations in large-scale software systems. The resulting works have been published in top-tier software engineering venues, including the Automated conference on Software Engineering (ASE), Partial Evaluation and Program Manipulation (PEPM) and the Journal of Automated Software Engineering (JASE). I have also been awarded the prestigious ACM SIGSOFT Distinguished Paper Award for my work on code clone refactoring.

Presently, as a Postdoctoral Researcher at Technische Universität Darmstadt, my work spans multiple research areas including API misuses, software cybersecurity and AI-based coding assistance. I manage a team of PhD students and student helpers to investigate and improve the state of the art in AI based coding assistance. I am also involved in maintaining the CogniCrypt project, advancing the state of the art in API misuse detection and software security. My contributions here have been published in various top-tier software engineering venues. I also actively maintain several national and international collaborations including with the University of Paderborn, University of Brasilia and Nanjing University, China. I also have experience in acquiring external funding, having helped author bulk of two successful grant proposals to the DFG. In terms of organizational and managerial activities, my experience serving on the hiring committee at Technische Universität Darmstadt, the directorate of Crossing SFB (\href[]{https://www.crossing.tu-darmstadt.de/crc_1119/people_crossing/directorate/index.en.jsp}{Link}) and the female student mentoring program (\href[]{https://www.crossing.tu-darmstadt.de/outreach/female_student_mentoring_and_networking/index.en.jsp}{Link}) have provided me with valuable insights into administrative processes in an academic setting. 

As an educator, I have honed my didactic skills through teaching various courses. I designed a lecture on concepts of programming languages, prepared the slides, recorded videos and offered the course multiple times both in person and online. The lectures are available online at the \href[]{https://www.youtube.com/playlist?list=PLjkB5k_lfPhTNUVNiGb_PeuioC-XV5kS5}{TU Darmstadt's STG lecture channel}. I also designed a novel course on AI based coding assistance where the end deliverable from the students is a scientific paper - \href[]{https://stg-tud.github.io/AI4CI/}{Course Website}. Several of the submissions from the course have been published at international workshops like CAIN and AISTA. 

I have a successful track record of industrial software development and client management, as demonstrated by my previous role as an IT Consultant/Software Architect at Itemis AG. Here, I worked closely with clients in the automotive and embedded systems industries (specifically Canon at Eindhoven, Netherlands and Siemens at Chennai, India) to design bespoke domain-specific languages and modelling tools. This hands-on industry experience enhanced my capacity to generate research results beneficial to both industry and society, a focus I believe is shared by the Software Technology group at Radboud University.

In terms of setting up my own research line, I have a proven record of leading research teams, being a significant part of acquiring external funding, and successfully pursuing novel research directions. Currently, I am particularly passionate about the intersection of traditional software development and innovative AI technologies.

I am thrilled at the prospect of contributing to Radboud University's pioneering work in software technology. I appreciate your consideration of my application and look forward to the opportunity of discussing my candidacy further.

\makeletterclosing

\end{document}