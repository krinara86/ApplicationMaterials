\documentclass[12pt,a4paper,sans]{moderncv}
\moderncvstyle{banking}
\moderncvcolor{blue}

\usepackage[utf8]{inputenc}
\usepackage[scale=0.75]{geometry}
\AtBeginDocument{
  \hypersetup{
    colorlinks=true,
    linkcolor=blue,
    filecolor=magenta,      
    urlcolor=cyan,
}
}

\name{Krishna}{Narasimhan}

\begin{document}

\recipient{Hiring manager}{Siemens\\Berlin, Germany}
\date{\today}
\opening{Dear Hiring manager of Siemens}
\closing{Sincerely,}

\makelettertitle

I am writing to express my profound interest in a Senior Software Engineer position at Siemens. I am an enthusiastic software developer and computer scientist with a strong background in model based tooling in the area of embedded systems. I have a proven track record of developing and maintaining large-scale software systems. 

Having earned a Ph.D. in Computer Science from Goethe University, I have cultivated extensive expertise in software analysis and transformation. My doctoral research was primarily devoted to innovating new techniques for automated refactoring of software clones and the automatic migration of data representations within large-scale software systems. My consequential works have received recognition and have been published in premier software engineering platforms. In acknowledgment of my work on software refactoring, I was honored with the prestigious ACM SIGSOFT Distinguished Paper Award - \href{https://ase-conferences.org/olbib/}{Link to ASE best papers}. I strongly adhere to the belief that software tools, regardless of whether they are research prototypes or industrial-grade products, ought to be user-friendly and provide a pleasant user experience. I have substantiated this through my rigorous evaluative approach; to verify the efficacy of my tools, I sent the tool-refactored versions of code as pull requests to high-quality repositories maintained by industry-leading companies like Google and Facebook. The merge status of these pull requests served as a valuable metric in validating my approach.

I have a successful track record of industrial software development and client management, as apparent from my previous role as an IT Consultant/Software Architect at Itemis AG. I worked closely with automotive and embedded systems clients (specifically Canon in Eindhoven, Netherlands, and Siemens in Chennai, India) to design bespoke domain-specific languages and modeling tools. This hands-on industry experience enhanced my ability to generate research results beneficial to industry and society, a focus I believe would be important for Siemens.

As a Postdoctoral Researcher at Technische Universität Darmstadt, my work spans multiple research areas, including API misuses, software cybersecurity, and AI-based coding assistance. I manage a team of Ph.D. students and student helpers to investigate and improve the state of the art in AI-based coding assistance. My research responsibilities also include maintaining the CogniCrypt project and advancing the state of the art in API misuse detection and software security. My contributions as a post-doc have been published in various top-tier software engineering venues. I actively maintain several national and international collaborations, including with the University of Paderborn, Germany, the University of Brasilia, Brazil, and Nanjing University, China. In terms of organizational and managerial activities, my experience serving on the hiring committee at Technische Universität Darmstadt, the directorate of Crossing SFB (\href[]{https://www.crossing.tu-darmstadt.de/crc_1119/people_crossing/directorate/index.en.jsp}{Link}) and the female student mentoring program (\href[]{https://www.crossing.tu-darmstadt.de/outreach/female_student_mentoring_and_networking/index.en.jsp}{Link}) have provided me with valuable insights into the administrative processes in a research setting. 



I am thrilled at the prospect of contributing to Siemens's pioneering work in systems engineering and research. I appreciate your consideration of my application and look forward to discussing my candidacy further.

\makeletterclosing

\end{document}