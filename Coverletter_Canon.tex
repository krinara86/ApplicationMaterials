\documentclass[12pt,a4paper,sans]{moderncv}
\moderncvstyle{banking}
\moderncvcolor{blue}

\usepackage[utf8]{inputenc}
\usepackage[scale=0.75]{geometry}
\AtBeginDocument{
  \hypersetup{
    colorlinks=true,
    linkcolor=blue,
    filecolor=magenta,      
    urlcolor=cyan,
}
}

\name{Krishna}{Narasimhan}

\begin{document}

\recipient{Ales Lacen}{Canon Printing Production\\Munich, Germany}
\date{\today}
\opening{Dear Ales Lacen}
\closing{Sincerely,}

\makelettertitle

I am writing to express my profound interest in a Embedded Software and Language developer position at Canon Printing production. I am an enthusiastic language engineer, software developer and computer scientist with a strong background in software tooling. I have a proven track record of developing and maintaining large-scale software systems. I am passionate about the intersection of traditional model-based software development and innovative AI technologies. I am thrilled at the prospect of contributing to Canon Printing production's pioneering work in software technology.

I have a successful track record of industrial software development and client management, as apparent from my previous role as an IT Consultant/Software Architect at Itemis AG. I worked closely with automotive and embedded systems clients (specifically Canon in Venlo, Netherlands, and Siemens in Chennai, India) to design bespoke domain-specific languages and modeling tools. This hands-on industry experience enhanced my ability to generate research results beneficial to industry and society, a focus I believe would be important for Canon Printing production.

Having earned a Ph.D. in Computer Science from Goethe University, I have cultivated extensive expertise in software analysis and transformation. My doctoral research was primarily devoted to innovating new techniques for automated refactoring of code clones and the automatic migration of data representations within large-scale software systems. My consequential works have received recognition and have been published in premier software engineering platforms such as the Automated Conference on Software Engineering (ASE), Partial Evaluation and Program Manipulation (PEPM), and the Journal of Automated Software Engineering (JASE). In acknowledgment of my work on code clone refactoring, I was honored with the prestigious ACM SIGSOFT Distinguished Paper Award - \href{https://ase-conferences.org/olbib/}{Link to ASE best papers}. I strongly adhere to the belief that developer tools, regardless of whether they are research prototypes or industrial-grade products, ought to be user-friendly and provide a pleasant user experience. I have substantiated this through my rigorous evaluative approach; to verify the efficacy of my tools, I sent the tool-refactored versions of code as pull requests to high-quality repositories maintained by industry-leading companies like Google and Facebook. The merge status of these pull requests served as a valuable metric in validating my approach.

As a Postdoctoral Researcher at Technische Universität Darmstadt, my work spans multiple research areas, including API misuses, software cybersecurity, and AI-based coding assistance. I manage a team of Ph.D. students and student helpers to investigate and improve the state of the art in AI-based coding assistance. My research responsibilities also include maintaining the CogniCrypt project and advancing the state of the art in API misuse detection and software security. A notable contribution in the area of API misuses includes jGuard, a DSL extension for Java enabling API usage annotation.  jGuard's unique ability to translate its code into standard Java code, where misuses are coded as exceptions, bypasses the need for external tools, mitigating misuse by design. My contributions as a post-doc have been published in various top-tier software engineering venues. I actively maintain several national and international collaborations, including with the University of Paderborn, Germany, the University of Brasilia, Brazil, and Nanjing University, China. In terms of organizational and managerial activities, my experience serving on the hiring committee at Technische Universität Darmstadt, the directorate of Crossing SFB (\href[]{https://www.crossing.tu-darmstadt.de/crc_1119/people_crossing/directorate/index.en.jsp}{Link}) and the female student mentoring program (\href[]{https://www.crossing.tu-darmstadt.de/outreach/female_student_mentoring_and_networking/index.en.jsp}{Link}) have provided me with valuable insights into the administrative processes in an academic setting. 

I am thrilled at the prospect of contributing to Canon Printing production's pioneering work in software technology. I appreciate your consideration of my application and look forward to discussing my candidacy further.

\makeletterclosing

\end{document}